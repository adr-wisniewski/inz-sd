\chapter{Instalacja}
\label{instalacja} % Always give a unique label

\section{Pliki konfiguracyjne}

\subsection{jdbc.properties}

Plik konfiguracyjny definiuj�cy po��czenie z baz� danych. Zawiera nast�puj�ce parametry:

\begin{description}

\item[jdbc.driverClassName] -- nazwa klasy do po��czenia z baz� danych,
\item[jdbc.url] -- adres url bazy danych,
\item[jdbc.username] -- nazwa u�ytkownika w bazie danych,
\item[jdbc.password] -- has�o u�ytkownika w bazie danych.

\end{description}

\subsection{messages.properties}

Plik zawieraj�cy tre�ci komunikat�w wy�wietlanych w aplikacji. Daje to mo�liwo�� zmiany w dowolnym momencie tre�ci dowolnego komunikatu, bez ponownej kompilacji aplikacji. Ka�dy komunikat jest postaci klucz, warto��.

\section{Skrypty SQL}

\subsection{ddl.sql}

Skrypt SQL tworz�cy wszystkie obiekty w bazie danych: tabele, perspektywy, wyzwalacze, pakiety, procedury sk�adowane, indeksy.

\subsection{inserts.sql}

Skrypt SQL wype�niaj�cy tabele przyk�adowymi danymi.

\section{Zainstalowanie aplikacji}

\subsection{Zainstalowanie aplikacji na serwerze Tomcat przy pomocy Tomcat Manager}

Po uruchomieniu aplikacji Tomcat Manager, w zak�adce \texttt{Deploy} nale�y wybra� plik \texttt{ServiceDesk.jar}, nast�pnie dokona� potwierdzenia poprzez klikni�cie w~przycisk \texttt{Deploy}.
