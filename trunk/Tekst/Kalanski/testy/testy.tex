\documentclass[a4paper, oneside, 11pt]{report}
\usepackage[a4paper, vmargin=1.0cm, top=0.5cm, bottom=2.0cm, headsep=1.0cm, nofoot ]{geometry}
\usepackage[utf8]{inputenc}      			% kodowanie literek
\usepackage[polish]{babel}				% reguły dla języka polskiego
\usepackage[OT4]{fontenc}				% ładne czcionki w PDFie
\usepackage{polski}					% polski
\usepackage{indentfirst} 				% wcięcia pierwszego akapitu
\usepackage{anysize}
\usepackage{colortbl}
% \usepackage{wrapfig} 					% oblewanie rysunków tekstem
\usepackage[pdftex]{graphicx} 				% dołączanie obrazków - tryb pdftex

\definecolor{tableheadercolor}{rgb}{.7,.75,1}
%\definecolor{tablerowcolor}{rgb}{.9,.95,1}
\definecolor{tablerowcolor}{rgb}{1,1,1}

\newenvironment{zenumerate}{\newcounter{zitem}\setcounter{zitem}{0}}
{}
\newcommand\zitem{\refstepcounter{zitem}\thezitem. }

\makeindex

\title{Service Desk}
\author{Piotr Kalański}

\selectlanguage{polish} 		% wybierz polski

\begin{document}

\maketitle

\tableofcontents

\chapter{Testy funkcjonalne}

% BEGIN template TEST CASE
\section{ID : Nazwa}
\subsection*{Cel}
%tutaj podaj cel
\subsection*{Warunki wstępne}
%tutaj podaj warunki wstępne
\subsection*{Kroki}
%tutaj podaj kroki
\begin{itemize}
	\item Krok1
	\item Krok2
	\item Krok3
\end{itemize}
\subsection*{Oczekiwany rezultat}
%tutaj podaj oczekiwany rezultat
% END template TEST CASE

\section{IM.ADD : Dodanie incydentu}
\subsection*{Cel}
Test sprawdza dodanie nowego incydentu w systemie.
\subsection*{Warunki wstępne}
Użytkownik jest zalogowany.
\subsection*{Kroki}
\begin{itemize}
	\item Wybranie opcji: dodanie incydentu
	\item Podanie tematu oraz opisu incydentu
	\item Wybranie zapisz
\end{itemize}
\subsection*{Oczekiwany rezultat}
Użytkownik zostanie przekierowany na kartę incydentu oraz system wyświetli komunikat o sukcesie.

\section{IM.MYOPEN : Otwarte incydenty}
\subsection*{Cel}
Test sprawdza widok otwartych incydentów użytkownka.
\subsection*{Warunki wstępne}
Użytkownik jest zalogowany.
\subsection*{Kroki}
\begin{itemize}
	\item Wybranie opcji: moje otwarte incydenty
\end{itemize}
\subsection*{Oczekiwany rezultat}
System wyświetli listę incydentów o statusie OPEN, które mają zgłaszającego aktualnie zalogowanego użytkownika.

\section{IM.MYCLOS : Zamknięte incydenty}
\subsection*{Cel}
Test sprawdza widok zamkniętych incydentów użytkownka.
\subsection*{Warunki wstępne}
Użytkownik jest zalogowany.
\subsection*{Kroki}
\begin{itemize}
	\item Wybranie opcji: moje zamknięte incydenty
\end{itemize}
\subsection*{Oczekiwany rezultat}
System wyświetli listę incydentów o statusie CLOS, które mają zgłaszającego aktualnie zalogowanego użytkownika.

\section{IM.SEARCH : Wyszukiwanie incydentów}
\subsection*{Cel}
Test sprawdza wyszukiwanie incydentów wg zadanego kryterium.
\subsection*{Warunki wstępne}
Użytkownik jest zalogowany.
\subsection*{Kroki}
\begin{itemize}
	\item Wybranie opcji: wyszukiwanie incydentów
	\item Podanie kryterium wyszukiwania
	\item Wybranie szukaj
\end{itemize}
\subsection*{Oczekiwany rezultat}
System wyśwetli lsitę incydentów spełniajacych zadane kryterium.

\section{IM.CARD : Widok karty incydentu}
\subsection*{Cel}
Test sprawdza wejście na kartę incydentu
\subsection*{Warunki wstępne}
Użytkownik jest zalogowany
\subsection*{Kroki}
\begin{itemize}
	\item Wybranie opcji: wyszukiwanie incydentów
	\item Wybranie szukaj
	\item Kliknięcie w temat wybranego incydentu
\end{itemize}
\subsection*{Oczekiwany rezultat}
System wyświetli widok ze szczegółami na temat wybranego incydentu.

\section{IM.EDIT : Edycja danych incydentu}
\subsection*{Cel}
Test sprawdza możliwość edycji danych dotyczących incydentu
\subsection*{Warunki wstępne}
Użytkownik jest na karcie incydentu
\subsection*{Kroki}
\begin{itemize}
	\item Zmienienie tematu
	\item Zmienienie opisu
	\item Ustawienie priorytetu
	\item Ustawienie kategorii
	\item Wybranie serwisanta
	\item Ustawienie statusu
	\item Ustawienei pilności
	\item Ustawienie grupy wsparcia
	\item Wybranie zapisz
\end{itemize}
\subsection*{Oczekiwany rezultat}
System wyświetli komunikat o sukcesie

\section{IM.RESOLVE.FAIL : Rozwiązanie incydentu - brak danych}
\subsection*{Cel}
Test sprawdza walidację przy zmianie statusu na rozwiązany
\subsection*{Warunki wstępne}
Użytkownik jest na karcie incydentu
\subsection*{Kroki}
\begin{itemize}
	\item Zmiana statusu na: rozwiązany
	\item Ustawienie rozwiązania na puste
	\item Wybranie zapisz
\end{itemize}
\subsection*{Oczekiwany rezultat}
System wyświetli komunikat o braku rozwiązania

\section{IM.RESOLVE.SUCCESS : Rozwiązanie incydentu}
\subsection*{Cel}
Test sprawdza funkcjonalność rozwiązania incydentu
\subsection*{Warunki wstępne}
Użytkownik jest na karcie incydentu
\subsection*{Kroki}
\begin{itemize}
	\item Zmiana statusu na: rozwiązany
	\item Wpisanie rozwiązania
	\item Wybranie zapisz
\end{itemize}
\subsection*{Oczekiwany rezultat}
System wyświetli komunikat o sukcesie oraz ustawi datę rozwiązania na atkualną datę.

\section{IM.CLOSE.FAIL : Zamknięcie incydentu - brak danych}
\subsection*{Cel}
Test sprawdza walidację przy zmianie statusu na zamknięty
\subsection*{Warunki wstępne}
Użytkownik jest na karcie incydentu
\subsection*{Kroki}
\begin{itemize}
	\item Zmiana statusu na: zamknięty
	\item Ustawienie atrybutów na wartości puste
	\item Wybranie zapisz
\end{itemize}
\subsection*{Oczekiwany rezultat}
System wyświetli komunikat o braku poszczególnych atrybutów

\section{IM.CLOSE.SUCCESS : Zamknięcie incydentu}
\subsection*{Cel}
Test sprawdza pomyślne zamknięcie incydentu
\subsection*{Warunki wstępne}
Użytkownik jest na karcie incydentu
\subsection*{Kroki}
\begin{itemize}
	\item Zmiana statusu na: zamknięty
	\item Uzupełnienie wszystkich danych: priorytet, wpływ, kategoria, serwisant, pilność, grupa wsparcia, rozwiązanie.
	\item Wybranie zapisz
\end{itemize}
\subsection*{Oczekiwany rezultat}
System wyświetli komunikat o sukcesie oraz ustawi datę zamknięcia na atkualną datę.

\section{IM.ADD.P : Dodanie problemu z karty}
\subsection*{Cel}
Test sprawdza funkcjonalność dodania problemu z karty incydentu
\subsection*{Warunki wstępne}
Uytkownik jest na karcie incydentu
\subsection*{Kroki}
\begin{itemize}
	\item Wybranie: Dodaj problem
	\item Podanie tematu oraz opisu problemu
	\item Wybranie zapisz
\end{itemize}
\subsection*{Oczekiwany rezultat}
System wyświetli kartę nowo dodanego problemu oraz wyświetli komunikat o sukcesie

%\begin{thebibliography}{99}

%\end{thebibliography}
\end{document}


